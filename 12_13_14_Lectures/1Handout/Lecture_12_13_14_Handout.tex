\documentclass[12] {article}
\usepackage{setspace}
\usepackage{amssymb,amsmath}
\usepackage[utf8x]{inputenc}


\begin{document}
\onehalfspace

\title{Lecture 12-13-14}
\date{}
\maketitle



\section{Basic Definition and a simple example}

Let $x = (x_1, x_2, \ldots, x_{T})$ denote some time series data set. Assume we have a statistical model for $X$---parameterized by $\theta$---and characterized by a p.d.f. $f(x | \theta)$.\\


\noindent \textbf{Definition} \textbf{(Likelihood):} The likelihood of data $x$ at parameter $\theta$ is defined as:
\[\mathcal{L}(\theta; x) \equiv f(x|\theta). \]

\noindent The likelihood function (corresponding to $x$) refers to the map $\mathcal{L}(\cdot; x): \Theta \rightarrow \mathbb{R}_{+}$. \\

\noindent Intuitively, $\mathcal{L}(\cdot; x)$ tells us how likely is the data that we observe at each different value of $\theta$. The log-likelihood function is simply defined as the natural logarithm of the likelihood: $\ln f(x | \theta)$.\\

\noindent \textbf{Definition} \textbf{(Maximum Likelihood Estimator):} The Maximum Likelihood (ML) Estimator of $\theta$ given data $x$ is defined as any value of $\theta \in \Theta$ that maximizes the likelihood; that is:
\[  \widehat{\theta} \in \textrm{argmax}_{\theta} \: \mathcal{L}(\theta; x). \]

ML estimation is popular because it generates consistent estimators for the parameters of interest (under very general conditions).\\

\noindent \textbf{Example for i.i.d. data:} Suppose that we observe $n$ i.i.d. realizations $(y_1,y_2, \ldots, y_n)$ from the following statistical model $y_i \in \{0,1\}$ and
\[ P_{\theta}(Y_i= 1) = \theta.\]

\noindent In this case we can show that the likelihood function can be written as:

$$ \mathcal{L}(\theta, (y_1, \ldots, y_n) ) = \theta^{n_1} (1-\theta)^{n_0} $$

\noindent where $n_1$ is the number of 1s in the sample of size $n$ and $n_0$ is the number of zeros. The maximum likelihood estimator of $\theta$ is the value that solves the problem:

\[  \max_{\theta}  \theta^{n_1} (1-\theta)^{n_0}.  \]

\noindent This is the same as maximizing:

\[  \max_{\theta} \ln \left(  \theta^{n_1} (1-\theta)^{n_0} \right)  \]

\noindent since the logarithm is a monotone function. Expanding the logarithm we get:

\[  \max_{\theta}   n_1 \ln \theta + n_0 \ln (1-\theta)   \]

\noindent which has the first-order conditions:

\[ \frac{n_1}{\theta} + \frac{n_0}{1-\theta}  = 0.\]

\noindent Solving for $\theta$ in the equation above we get that:

\[ \widehat{\theta}_{\textrm{ML}} = \frac{n_1}{n_1+ n_0}. \]

\noindent We can use the LLN for independent data discussed in class to show that 

$$ \widehat{\theta}_{\textrm{ML}} \overset{p}{\rightarrow} E_{\theta}[  Y_i = 1  ] = \theta.$$
 
\newpage 
 
 \section{ML Estimation for the MA(1)/AR(1) model}
 
We will need two things to compute a Maximum Likelihood estimator: a likelihood function and some computer program to optimize its value. We will try to get a sense of what these requirements entail in the context of the MA(1)/AR(1) models. 

\subsection{Gaussian MA(1)}
We will first discuss the Maximum Likelihood estimation of the MA(1) model: 

\begin{equation} \label{equation:MA1}
X_t = \mu + \epsilon_t + \theta \epsilon_{t-1} 
\end{equation}

\noindent In order to get a likelihood we will assume that the shocks are i.i.d. and:

$$ \epsilon_{t} \sim N(0, \sigma^2).$$

\noindent The parameters of this model are $(\mu, \theta, \sigma^2)$. It is not difficult to verify that for any $t$:

$$   
\begin{pmatrix}
X_{t-1} \\
X_{t}
\end{pmatrix} \sim 
\mathcal{N} 
\left( 
\begin{pmatrix}
\mu \\
\mu
\end{pmatrix}, \\
\sigma^2 \begin{pmatrix}
1+ \theta^2 & \theta \\
\theta & 1+ \theta^2
\end{pmatrix}
\right).
$$
\noindent Note also that $X_{t}$ is independent of the collections $X_{t-2}, \ldots X_{1}$ and  $X_{t+2}, \ldots X_{T}$. This means the likelihood of $(X_1,\ldots, X_{T})$ at parameters $(\mu, \theta, \sigma^2)$
is given by:
\begin{equation}\label{equation:LikelihoodMA1}
\begin{pmatrix}
X_1 \\ X_2 \\ \vdots \\ X_{T-1} \\ X_{T} 
\end{pmatrix}
\sim \mathcal{N} \left( 
\begin{pmatrix}
\mu \\ \mu \\ \vdots \\ \mu \\ \mu
\end{pmatrix}, 
\sigma^2
\begin{pmatrix}
1 + \theta^2 & \theta & \ldots & 0 & 0 \\
\theta & 1+\theta^2 & \ldots & 0 &0 \\
\vdots & \vdots & \vdots & \vdots & \vdots \\
0 & 0 & \ldots & 1+\theta^2 & \theta \\
0 & 0 & \ldots & \theta & 1  + \theta^2
\end{pmatrix}
\right).
\end{equation}

The formula above implies that in order to compute the likelihood function of the MA(1) model, all we need is the p.d.f of a multivariate normal distribution where the vector of means and the covariance matrix are functions of $\mu, \theta$, and $\sigma^2$.\footnote{See wikipedia for the formula of the p.d.f. of a multivariate normal random vector.}  We can easily write a Matlab function that evaluates the likelihood of the MA model by using the matlab command $\texttt{y = mvnpdf(R,MU,SIGMA)}$.\footnote{This command evaluates the multivariate normal p.d.f. for a row vector R using the vector of means MU and the covariance matrix SIGMA. This command only works if $X$ is not very large. } \\

\subsection{Tricks to evaluate the likelihood of an MA(1) model}

Oftentimes, there are some tricks to evaluate the likelihood in time series models. Note that the definition of conditional density  implies that:
$$ f(X_1, X_2, \ldots, X_T)  = f(X_T | X_{T-1}, X_{T-2}, \ldots, X_{1}) f(X_{1}, X_{2}, \ldots, X_{T-1}) $$ 
Applying the definition one conditional density one more time we have that
$$ f(X_{1}, X_{2}, \ldots, X_{T})$$
equals
$$f(X_T | X_{T-1}, X_{T-2}, \ldots, X_{1}) f(X_{T-1} | X_{T-2}, X_{T-3}, \ldots, X_{1}) f(X_{1}, X_{2}, \ldots, X_{T-2}).$$
\noindent Doing this recursively we get that:
$$ f(X_1, X_2, \ldots, X_{T}) = \Pi_{t=2}^{T} f(X_t | X_{t-1}, X_{t-2}, \ldots, X_{1}).$$

\noindent Why is this useful? Well, MA(1) and AR(1) models make simplifications on the conditional distributions $f(X_t | X_{t-1}, \ldots, X_{1})$. Such simplifications often lead to simpler expressions for the likelihood in (\ref{equation:LikelihoodMA1}). For instance, note that for an MA(1) model
$$ X_{t} | X_{t-1}, X_{t-2}, \ldots, X_1  \sim X_{t} | X_{t-1} \sim \mathcal{N}( \mu + \rho(X_{t-1}-\mu) \: , \: (1-\rho^2) \gamma_0 ),$$

\noindent with $\rho \equiv \gamma_1 / \gamma_0, \gamma_1 \equiv \theta \sigma^2, \gamma_0 \equiv \sigma^2(1+ \theta^2)$. This means that we can easily write the log-likelihood of $(X_2, X_3, \ldots, X_{T})$
$$\ln f(X_2, \ldots, X_{T} | (\mu, \theta, \sigma^2))$$
as 
\begin{equation} \label{equation:LikelihoodMA1-2}
c - \frac{T-1}{2} \ln (1-\rho^2) - \frac{T-1}{2} \ln (\gamma_0) - \frac{1}{2 (1-\rho^2) \gamma_0 } \sum_{t=2}^{T} (X_t - \mu - \rho (X_{t-1}-\mu))^2. 
\end{equation}

\noindent You might not realize it now, but evaluating (\ref{equation:LikelihoodMA1-2}) is computationally less demanding than evaluating (\ref{equation:LikelihoodMA1}). I will now argue, that the trick is also useful to derive a closed-form expression for the ML estimators (which are typically not available). \\

\noindent Note that there is a one-to-one mapping between $(\mu, \theta, \sigma^2)$ and $(\mu, \rho, \gamma_0)$ (by virtue of the results presented in Lecture 10). 

Assume, for the sake of exposition, that we have $\mu=0$ and let us derive the ML estimators of $(\rho, \gamma_0)$. Note that the first-order conditions for $\gamma_0$ are given by:
$$-\frac{T-1}{2 \gamma_0} + \frac{1}{2(1-\rho^2) \gamma_0^2} \sum_{t=2}^{T} (X_t - \rho X_{t-1})^2 = 0$$
\noindent This implies that the ML estimators of $(\widehat{\rho}, \widehat{\gamma}_0)$---whatever they are---need to satisfy:
\begin{equation} \label{equation:FOCgamma}
\widehat{\gamma}_0 = \frac{1}{(1-\widehat{\rho}^2)} \frac{1}{T-1} \sum_{t=2}^{T} (X_t - \widehat{\rho} X_{t-1})^2.
\end{equation}

\noindent I will now use the condition above to figure out the expression for $\widehat{\rho}$. This derivative is a little bit more complicated, but it can be shown that the first order condition is given by:
$$ \frac{T-1}{2} \frac{2 \widehat{\rho}}{1-\widehat{\rho}^2} - \frac{2 \widehat{\rho} (T-1) }{2\widehat{\gamma}_0 (1-\widehat{\rho}^2)^2 }  \frac{1}{T-1} \sum_{t=2}^{T} (X_t - \widehat{\rho} X_{t-1})^2 + \frac{2}{2 (1-\widehat{\rho}^2) \widehat{\gamma}_0} \sum_{t=2}^{T} (X_t - \widehat{\rho} X_{t-1}) X_{t-1}$$   

\noindent equals 0. This means that the F.O.C. is satisfied if and only if:

$$ \frac{(T-1) \widehat{\rho}}{1-\widehat{\rho}^2} - \frac{ \widehat{\rho} (T-1) }{ 1-\widehat{\rho}^2 }  + \frac{1}{ (1-\widehat{\rho}^2) \widehat{\gamma}_0} \sum_{t=2}^{T} (X_t - \widehat{\rho} X_{t-1}) X_{t-1} = 0.$$

\noindent Which implies that:
\begin{equation} \label{equation:FOCrho}
\widehat{\rho} = \sum_{t=2}^{T} X_{t}X_{t-1} / \sum_{t=2}^{T} X_{t-1}^2.  
\end{equation}

\noindent How do we get the estimators of $\theta$ and $\sigma^2$? Here is how:

\begin{enumerate}
\item Compute $\widehat{\rho}$ from (\ref{equation:FOCrho}) and $\widehat{\gamma}_0$ from (\ref{equation:FOCgamma}).
\item Define $\widehat{\gamma}_1$ as $\rho \widehat{\gamma}_0$ 
\item Use the derivations in Lecture 10 to get $\widehat{\theta}$ and $\widehat{\sigma}^2$ from $\widehat{\gamma}_0$ and $\widehat{\gamma}_1$. 
\end{enumerate}
\noindent Compare the output with Matlab’s \texttt{arima} package. Are they the same?

\newpage

\subsection{Gaussian AR(1) model}
   
We will now derive the likelihood of the Gaussian AR(1) model:

\[ X_t  = \mu + \phi X_{t-1} + \epsilon_{t}, \quad \epsilon_{t} \sim \mathcal{N}(0,\sigma^2) \] 

\noindent It is not difficult to show, that the Gaussian AR(1) model has the following causal linear process representation:

\[ X_t = \frac{\mu}{1-\phi} + \sum_{j=0}^{\infty} \phi^j \epsilon_{t-j} \]

\noindent which suggests that:

\begin{equation}
\begin{pmatrix}
X_1 \\ X_2 \\ \vdots \\ X_{T-1} \\ X_{T} 
\end{pmatrix}
\sim \mathcal{N} \left( 
\begin{pmatrix}
 \frac{\mu}{1-\phi} \\ \frac{\mu}{1-\phi} \\ \vdots \\ \frac{\mu}{1-\phi} \\ \frac{\mu}{1-\phi}
\end{pmatrix}, 
\frac{\sigma^2}{1-\phi^2} 
\begin{pmatrix}
1 & \phi & \ldots & \phi^{T-2} & \phi^{T-1} \\
\phi & 1 & \ldots & \phi^{T-3} & \phi^{T-2} \\
\vdots & \vdots & \vdots & \vdots & \vdots \\
\phi^{T-2} & \phi^{T-1} & \ldots & 1 & \phi \\
\phi^{T-1} & \phi^{T-2} & \ldots & \phi & 1 
\end{pmatrix}
\right).
\end{equation}

\subsection{Tricks to evaluate the likelihood of the AR(1) model}  
   
We use the same trick that we used for the MA(1) model. Note that the AR(1) model implies that:
$$  X_t | X_{t-1} \sim \mathcal{N}( \mu + \phi X_t , \sigma^2 ).$$
The log-likelihood of $(X_2, X_3, \ldots, X_T)$ is then given by:
$$ c - \frac{T-1}{2} \ln (\sigma^2) - \frac{1}{2 \sigma^2} \sum_{t=2}^{T} (X_t - \mu - \phi X_{t-1})^2.$$
Once again, and for the sake of exposition, we set $\mu=0$. The F.O.C for $\phi$ in this problem imply that:
$$ -\frac{1}{2 \sigma^2} \sum_{t=2}^{T} 2(X_t - \widehat{\phi} X_{t-1}) X_{t-1} = 0, $$
which implies that
$$ \widehat{\phi} = \sum_{t=2}^{T} X_{t}X_{t-1} / \sum_{t=2}^{T} X_{t-1}^2.  $$             
The first order conditions for $\sigma^2$ give:
$$ -\frac{T-1}{2 \widehat{\sigma}^2} - \frac{1}{2 \widehat{\sigma}^4}(X_t - \widehat{\phi} X_{t-1})^2 = 0,$$
which imply that:
$$ \widehat{\sigma}^2 = \frac{1}{T-1} \sum_{t=2}^{T} (X_t - \widehat{\phi} X_{t-1})^2  $$

\end{document}